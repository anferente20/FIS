\chapter{Caso de Estudio}

\section{Introducción}

En los últimos cincuenta años se ha desarrollado una revolución tecnologica de una manera tan marcada que ha obligado al mundo a adaptarse y evolucionar a un ritmo vetigínoso; este cambio es algo que afecta a todos los miembros activos de la sociedad, en especial a las empresas, pues han visto como su funcionamiento es cada vez más complejo, puesto que cada vez están en contacto con un número mayor de clientela, lo cual las obliga a buscar ser más productivas e implementan una mayor cantidad de personal. Por todo esto, en los últimos años las empresas han empezado a buscar soluciones tecnológicas que permitan simplificar los procesos que se desarrollan dentro de ellas y lograr tal objetivo de productividad.
 
\section{Descripción del problema}

Una empresa de cines, ubicada en Bogotá, busca desarrollar un software que le permita administrar su funcionamiento. Dicho software debe estar en la capacidad de gestionar la boletería de todos los cinemas que tiene a lo largo de la ciudad para cada una de las diversas funciones ofrecidas en las distintas horas del día en tareas como compra y reserva de boletas, al igual que las ventas de confiteria de cada cinema. Además estará en la capacidad de llevar un registro y monitoreo de los empleados y el manejo de insumos y maquinaria de confitería que estén a disposición de cada cinema.

Durante el siguiente libro se realizará un estudio del problema planteado, buscando encontrar una solución eficiente mediante la implementación de la ingeniería de software.

\section{Objetivo General}
Generar una solución de software completa que permita administrar y monitorear las actividades principales de una cadena de cines, mediante el uso de metodologías y fundamentos básicos de ingeniería de software.

\subsection{Objetivos específicos}

\begin{enumerate}
	\item Gestionar el manejo de boletería y funciones para cada uno de los cines a lo largo de la ciudad de Bogotá D.C. incluyendo tanto la compra y reserva de boletas como selección de asientos de las funciones.
	\item Administar el manejo de los insumos en la sección de confitería para cada uno de los cinemas de la ciudad, mediante el uso de bases de datos.
	\item Desarrollar un cliente de uso para el público general y uno diferente de uso para personal administrativo.
	\item Implementar un registro de los empleados que tiene la empresa en cada uno de los cinemas. 
\end{enumerate}

\section{Alcances}
Se implementará un cliente de uso para público general que podrá realizar las siguientes actividades:
\begin{itemize}
	\item Compra y reserva de boletas.
	\item Selección de asientos.
	\item Visualización de peliculas en cartelera y próximos estrenos.
	\item Gestión de suscripción especial.
	\item Exposición y compra de confiteria.
	\item Generación de tickets de pago.
\end{itemize}

Se implementará un cliente de uso para personal administrativo que podrá realizar las siguientes actividades:
\begin{itemize}
	\item Gestionamiento de cartelera, próximos estrenos y confiteria.
	\item Asignación de funciones.
	\item Manejo del inventario de confiteria.
	\item Gestionamiento de personal a nivel de registro.
\end{itemize}


\section{Limitaciones}
\begin{itemize}
\item El cliente público se limitará a la emisión de tickets de pago, no contará con la posibilidad de un módulo de pago.

\item El cliente de personal administrativo no contará con el manejo de nómina, solamente se restringirá al registro de personal de trabajo.
\end{itemize}


